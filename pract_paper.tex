\documentclass[11pt]{article}
\usepackage{amsmath}
\usepackage{mathbbol}
\usepackage{graphicx}
\usepackage{hyperref}

\graphicspath{{./}}
\nocite{*}

\title{INFORME DE LA PRÁCTICA PROFESIONAL 
FACULTAD DE MATEMÁTICA Y COMPUTACIÓN}
\author{
    \bf {Autor: Adrian Hernández Santos}\\
    \bf {Grupo: 211}\\
    \bf {Carrera: Ciencia de la Computación}\\
    \bf {Curso: 2022}\\
    \bf {Tutor: Damian Valdés Santiago}\\
    \bf {Departamento Matemática Aplicada - MATCOM - UH}
}
\date{}

\normalsize
\begin{document}
\maketitle

\vfill
{
    \centering
    {
        \Large{\bf {Formulación de filtros de señales para identificar locutores auténticos e inauténticos}}
    }
}

\paragraph*{Resumen:} En este informe se describe el proceso de formulación de filtros de señales para caracterizar 
los locutores auténticos e inauténticos que pudieran intervenir en la misma, utilizando transformadas wavelet y 
sistemas de ecuaciones simbólicas y no lineales.

\clearpage

\section*{Introducción}
El proceso de análisis de datos, específicamente de señales de diversa índole, es necesario en casi todos los aspectos 
de la ciencia y la técnica. Dado a la utilización de equipos de cómputo en la gran mayoría de los procesos de la sociedad, 
se hace inminente un método eficaz y preciso para transformar esos datos en conocimiento que aporte información procesable.
Con el auge de la Inteligencia Artificial y el Aprendizaje de Máquina, es muy frecuente su uso. Pero dichos procesos requieren una cantidad 
de datos iniciales para garantizar una respuesta acertada, algo que no es posible en algunos contextos. Por esta razón 
se ha optado por desarrollar un enfoque basado en modelos de optimización para obtener una solución de una calidad similar 
pero reduciendo la cantidad de datos iniciales para modelar dicho problema. Llevando esta problemática específicamente 
a la detección de locutores en señales de audio, se propone una solucion utilizando un modelo basado en caracterización de 
dichas señales mediante filtros, dependientes de coeficientes de conocimientos únicos para un determinado conjunto de datos.

\clearpage

\section*{Descripción de la Institución}
\paragraph*{Departamento Matemática Aplicada, Facultad de Matemática y Computación, Universidad de la Habana:}
Grupo de investigación con soluciones a problemáticas utilizando Matemática Aplicada: procesos de optimización, ecuaciones diferenciales, métodos númericos, entre otros.


\clearpage

\section*{Descripción de las Actividades}
\paragraph*{Investigación:}
Durante esta fase, el objetivo fue la familiarización con el problema en cuestión: la necesidad de dado un conjunto de señales, determinar filtros que caractericen 
cualquier señal de un locutor en auténtico o inauténtico, con respecto a las señales de entrada. Utilizando la bibliografía relativa al problema se pudo dividir dicho problema en 
3 subproblemas mas pequeños, cuya unión da solución a la problemática inicial, utilizando el enfoque propuesto por {\it de Almeida, Guido} en \cite{ag2022} \cite{ag2021}:

\section*{Subproblema 1:}
Dado un tamaño \(N\), se deben hallar los filtros \(q[n]\) y \(p[n]\). Dichos filtros quedarían en función 
de \(K = N - (\frac{N}{2} - 1)\) coeficientes de conocimiento. Este proceso no devuelve directamente los valores de \(q\) y \(p\) reales, sino que formarán un sistema de 
ecuaciones no lineal con dichos filtros en función de los coeficientes de conocimiento \(K\), los cuales se aplicarán a la descomposición de las señales. El algoritmo 
para formular dichos filtros \(q\) y \(p\) es el siguiente:

\noindent {\section*{\underline{INICIO}}}

\paragraph*{Paso 1:}
Los \(q\) se espera que tengan energía unitaria, por tanto, preservando la energía de la señal original se tiene que:
\begin{equation}
    \sum_{k=0}^{N - 1} q^2 = 1
\end{equation}

\paragraph*{Paso 2:}
\(\frac{N}{2}\) momentos nulos son aplicados, utilizados en la familia {\it wavelet} Symlet:

\begin{equation}
    \sum_{k=0}^{N - 1} q_{k}(\frac{N}{2} + 1 + k)^b = 0
\end{equation}

donde \(b = \{1, 2, \cdots, \frac{N}{2} - 1\}\).

\paragraph*{Paso 3:}
\(\frac{N}{2} - 1\) restricciones de clase son impuestas:

\begin{equation}
    \sum_{k=0, j=0}^{N - 1} q_{k}K_{j} = 0
\end{equation}

Donde \(K_{j}\) son los coeficientes de conocimiento.

\paragraph*{Paso 4:}
Las ecuaciones previamente formuladas se agrupan formando un sistema de ecuaciones no lineal de \(N\) ecuaciones y \(N\) incógnitas, que corresponden a 
los filtros \(q\). Dicho sistema es simbólico, pues sus soluciones quedan en función de \(K = N * (\frac{N}{2} - 1)\) 
incógnitas que serían los coeficientes de conocimiento.

\paragraph*{Paso 5:}
Una vez obtenido \(q[n]\) en función de los \(K\) coeficientes de conocimiento podemos formular 
los \(N\) filtros \(p[n]\) mediante la forma:

\begin{equation}
    p_{k} = (-1)^{k-1}q_{N - k - 1}
\end{equation}

\noindent {\section*{\underline{FIN}}}

\section*{Subproblema 2:}
Teniendo los filtros \(q\) y \(p\) en función de los \(K\) coeficientes de conocimiento, se hace necesario obtener dichos coeficientes. Se tiene una entrada de \(M = N\) 
señales numéricas. Realizando el producto matricial \(As = r\), donde \(A[n \times n]\) representa una matriz de {\it Cuadratura de Filtros Espejados}, formulada a partir de 
\(q[n]\) y \(p[n]\); \(s[n]\) es una señal de entrada, con las muestras del contenido a filtrar; y \(r[n]\) la señal transformada. Dicho proceso es llamado {\it Algoritmo 
de Mallat} \cite{mallat}, el cual es un algoritmo de {\it Transformada Wavelet}. Consiste en una multiplicación de dicha matriz, por una señal de entrada, donde el resultado de concatenar 
los \(\frac{N}{2}\) índices pares de \(r[n]\) con los \(\frac{N}{2}\) índices impares corresponde al primer nivel de la {\it Transformada Wavelet Discreta} sobre \(s[n]\). Realizando este proceso \(\log_{2}{N}\) 
veces hasta que la señal de entrada tenga solamente un único coeficiente, obteniendo la transformada de dicha señal, donde en cada iteración se reformula la matriz \(A\), atendiendo al tamaño de la entrada 
del nivel correspondiente de la transformada. Para la solución de este subproblema se propone el siguiente algoritmo:

\noindent {\section*{\underline{INICIO}}}

\paragraph*{Paso 1:}
Formulando la matriz \(A\) en función de \(p\), \(q\) y \(M\) de la siguiente forma:

\begin{equation}
    \begin{pmatrix}
        q_0 & q_1 & q_2 & q_3 & q_4 & ... & q_{M - 1} & 0 & ... & 0\\ 
        p_0 & p_1 & p_2 & p_3 & p_4 & ... & p_{M - 1} & 0 & ... & 0\\ 
        0 & 0 & q_0 & q_1 & q_2 & ... & 0 & q_{M - 1} & ... & 0\\ 
        0 & 0 & p_0 & p_1 & q_2 & ... & 0 & p_{M - 1} & ... & 0\\ 
        ... & ... & ... & ... & ... & ... & ... & ... & ... & ...\\
        ... & ... & ... & ... & ... & ... & ... & ... & ... & ...\\
        q_2 & q_3 & ... & ... & q_{M - 1} & 0 &... & 0 & q_0 & q_1\\
        p_2 & p_3 & ... & ... & p_{M - 1} & 0 &... & 0 & p_0 & p_1\\
    \end{pmatrix}
\end{equation}

\paragraph*{Paso 2:}
Utilizando el {\it Algoritmo de Mallat} para hallar \(TWD{s[M]}\) {\it (Transformada Wavelet Discreta para \(s[M]\))}, realizando el producto matricial:

\begin{equation}
    A \times s = r
\end{equation}

Proceso que se repite \(\log_{2}N\) veces, donde luego de cada vez que efectua el producto el primer nivel de la 
\(TWD_{s}\) quedaría como la concatenación de los índices pares e impares del resultado \(r\) de la forma: \(r = \{r_{0}, r_{2}, r_{4}, \cdots, r_{N - 2}, r_{1}, r_{3}, r_{5}, \cdots, r_{N - 1}\}\), ya que se espera 
que la entrada \(N\) siempre sea par.

\paragraph*{Paso 3:}
Luego, cada nivel subsecuente de la \(TWD_{s}\) sería el resultado de aplicar el {\bf Paso 2} a los \(\frac{N}{2}\) resultados del nivel anterior, hasta llegar al nivel \(\frac{\log_{2}M}{2}\), con el cual se obtiene la transformada de 
nivel medio de la señal original \(s\).

\noindent {\section*{\underline{FIN}}}

\section*{Subproblema 3:}
Luego de tener las señales transformadas de nivel medio, se hace necesario caracterizar dichas señales. Para dicha caracterización se pueden utilizar distintas funciones que permitan apreciar las propiedades físicas de las señales 
transformadas: para este trabajo se han seleccionado {\it Skewness (Asimetría)} y {\it Kurtosis (Curtosis)}. Aplicando dicha caracterización de propiedades físicas a cada \(Z_{i}[.]\), que representan los resultados del proceso de descomposición 
con {\it Transformada Wavelet} en el nivel medio \(i = \frac{\log_{2}M}{2}\), se puede formular un \(\lambda\) para cada fenómeno físico a caracterizar, que permite filtrar dichos resultados y crear una especie de punto de ruptura para formar nuestra caracterización en locutores auténticos e 
inauténticos. Tomando los \(Z_{i}[.]\) que corresponden a las transformadas de frecuencias mayores a 1 kHz, se puede calcular una media \(\mu_{AF}\) y la desviación estándar \(\sigma_{AF}\). El siguiente algoritmo permite formular el segundo 
sistema de ecuaciones no lineal, que permite obtener el valor de los coeficientes de conocimiento \(K\):

\noindent {\section*{\underline{INICIO}}}
\paragraph*{Paso 1:}
Sea \(f = Skewness \) función que modela las propiedades asimétricas de una señal y \(Z_{i}[.]\) la transformada de nivel medio de \(s\) de altas frecuencias:

\begin{equation}
    \mu_{AF} = \frac{\sum_{i=1}^{2^{\log_{2}(M - J)} - 1} f(Z_{i}[.])}{2^{\log{2}(M - J)} - 1}
\end{equation}

\begin{equation}
    \sigma_{AF} = \frac{\sum_{i=1}^{2^{\log_{2}(M - J) - 1}}(f(Z_{i}[.]) - \mu )^2}{2^{\log{2}(M - J)} - 1}
\end{equation}

\begin{equation}
    \lambda_{kur} = \mu_{AF} + 2 \times \sigma_{AF}
\end{equation}

\begin{equation}
    \lambda_{skw} = \mu_{AF} + \sigma_{AF}
\end{equation}

donde \(M\) es el tamaño de la transformada \(Z\) de nivel medio y \(J\) la cantidad de muestras de \(Z\) con frecuencias menores que 1 kHz.

\paragraph*{Paso 2:}
En función de esos \(\lambda\), se pueden agrupar las señales \(Z\) de tal forma que se puedan filtrar los auténticos de los inauténticos:

\begin{equation}
    \begin{cases}
        {\bf Aut\acute{e}nticos} & \text {\(f(Z_{0}[.]) <= \lambda\)}\\
        {\bf Inaut\acute{e}nticos} & \text{\(f(Z_{0}[.]) > \lambda\)}\\
    \end{cases}
\end{equation}

\paragraph*{Paso 3:}
Para formular el segundo sistema no lineal, cuya solución serán los coeficientes \(K\), se toman las señales de entrada que pasan por el proceso de descomposición usando el filtro \(q\), 
se formula una ecuación de caracterización de conocimiento, o sea, el número de señales de entrada implica el número de ecuaciones que componen el segundo sistema no lineal y su solución: 
los coeficientes \(K\).

\paragraph*{Paso 4:}
Una vez solucionado el sistema no lineal del {\bf Paso 4} y obtenidos los \(N \times (\frac{N}{2} - 1)\) coeficientes de conocimiento \(K\), se sustituyen esos valores numéricos en las ecuaciones simbólicas \(q\), o sea 
\(q(k_{0}, k_{1}, \cdots, k_{N \times (\frac{N}{2} - 1) -1})\), cuyo resultado produciría los \(N\) coeficientes numéricos \(q_{i}\) de los filtros basados en conocimiento.

\noindent {\section*{\underline{FIN}}}

\clearpage

\section*{Conclusiones}
Durante la práctica laboral, se realizaron investigaciones y experimentos en el ámbito del procesamiento de señales, cuestiones que no se tratan en el plan de estudio básico de la carrera Ciencias de la Computación, 
pero que contienen un alto contenido teórico/práctico que es de amplio uso e importancia en muchos sectores de la ciencia y la técnica, que es a fin de cuentas, el objetivo principal en la formación de un estudiante. A pesar de que, 
por cuestiones de tiempo, solamente se pudo abarcar una solución del problema principal hasta el {\bf Subproblema 2}, el resto queda para una posterior etapa de prácticas laborales, donde se ampliará aún más sobre el tema. Se intentará 
mejorar las soluciones iniciales e incluso lograr una solución general para el mismo, utilizando un enfoque similar al del problema en cuestión. A pesar de esto, la investigación realizada ha aportado conocimientos extra de asignaturas como 
{\it Álgebra, Análisis Matemático, Matemática Numérica y Programación}, así como un acercamiento a problemáticas de la vida real y cómo se hayan soluciones prácticas a estas. 

\clearpage

\section*{Anexos}

\paragraph*{Bibliotecas utilizadas en la formulación del código:}
\begin{enumerate}
    \item \href{https://numpy.org/}{NumPy} 
    \item \href{https://www.sympy.org/}{SymPy} 
\end{enumerate}

\paragraph*{Transformada Symlet:}
Familia {\it wavelet} con amplio uso en el tratamiento de señales de voz, debido a que su forma refleja bastante bien las características principales de la voz.

\paragraph*{Transformada Wavelet:}
Una {\it wavelet} \cite{wavelet}, es una señal que tiene una duración limitada y un valor promedio de 0. La principal diferencia con la transformada de Fourier, para el análisis de señales son sus propiedades de asimetría e irregularidad, las cuales describen 
mucho mejor el comportamiento real de una señal, ya sea analógica o digital. Al seleccionar una {\it wavelet} de las distintas familias existentes, por su similitud con el tipo de señal que queremos analizar, se le pueden realizar cambios, puesto que esta puede trasladarse 
o escalar (comprimirse o expandirse) a lo largo de nuestra señal. Si definimos nuestra wavelet como \(\psi_{a, b}(t)\), siendo \(a\) y \(b\) los parámetros de escalamiento y desplazamiento respectivamente se tiene:

\begin{equation*}
    \psi_{a,b}(t) = \frac{1}{\sqrt{a}}\psi(\frac{t - b}{a})
\end{equation*}

Teniendo esto se puede definir la {\it Transformada Wavelet Continua} como el producto de una señal continua y su {\it wavelet} asociada en su versión desplazada y escalada de la forma:

\begin{equation*}
    TWC(a, b) = \frac{1}{\sqrt{a}}\int_{-\infty}^{\infty}x(t)\psi(\frac{t - b}{a})dt
\end{equation*}

donde \(x(t)\) es la señal continua.\\
La TWD se obtiene al discretizar los parámetros de desplazamiento y escalamiento dentro de la TWC. Generalmente los valores que se utilizan son:

\begin{equation*}
    \begin{cases}
        a=2^{-j}\\
        b=k2^{-j}
    \end{cases}
\end{equation*}

donde \(a\) es el escalamiento, \(b\) el desplazamiento y \(j, k\) valores enteros. Dado esto se tiene que la nueva {\it wavelet} queda de la forma:

\begin{equation*}
    \psi_{j, k}(t)= w^{\frac{j}{2}}\psi(2^{j}t - k); j, k \in \mathbb{Z}
\end{equation*}

Por lo que se define la TWD como:

\begin{equation*}
    TWD_{j, k} = 2^{\frac{j}{2}}\int_{-\infty}^{\infty}x(t)\psi(2^{j}t-k)dt
\end{equation*}

con \(x(t)\) una señal discreta. 

\paragraph*{Filtro de convolución (pasa-bajas o pasa-altas):}
Expresión cuya función principal es atenuar las señales de bajas(altas) frecuencias, para concentrarse en las de altas(bajas) frecuencias. El proceso de convolución esta dado por la transformación de 2 señales digitales en una tercera, basándose en ciertas 
características y con un objetivo específico, dependiendo de la transformada que se utilice en el proceso.

\paragraph*{Cuadratura de Filtros Espejados:}
Matriz de convolución asociada a los filtros pasa-bajas y pasa-altas, en la que cada par de filas contiene a dichos filtros, con un desfase de 2 columnas con respecto al par de filas anterior y rellenando todos los demás espacios vacíos con 0. Esta matriz de convolución, 
se utiliza para hacer {\it downsampling} a la señal de entrada.

\paragraph*{Algoritmo de Mallat:}
Stephane G. Mallat, en \cite{mallat}, plantea una forma de descomponer señales de una y dos dimensiones relacionadas a imágenes, utilizando {\it Transformadas Wavelet}, para facilitar el análisis 
de distintas propiedades de dichas imágenes. Este proceso se basa en una formulación, utilizando las bases ortonormales respectivas a una {\it wavelet}, para formular un sistema piramidal, en el cual utilizando un banco de filtros, se puede 
hacer {\it downsampling} de la señal original para su posterior clasificación, disminuyendo su resolución, pero sabiendo en todo momento el cambio de sus propiedades con respecto a una resolución anterior. Este proceso es válido incluso en sentido inverso, realizando un {\it upsampling}, para 
revertir el proceso, e incluso recrear dichas imágenes (señales), a partir de otras descomposiciones. En este informe y algunos de los citados, se utiliza una adaptación de ese algoritmo, para realizar la descomposición de una señal de entrada, para su posterior clasificación utilizando 
Asimetría ({\it Skewness}) y Curtosis ({\it Kurtosis}), para obtener un parámetro de clasificación basado en conocimiento y encontrar la solución al problema inicial.

\paragraph*{Skewness:}
Propiedad probabilística que clasifica una distribución en cuanto a su asimetría con respecto a su media. Su forma es muy similar a la distribución de Poisson \cite{skew}.

\paragraph*{Kurtosis:}
Propiedad probabilística que clasifica una distribución con respecto a que tan concentrada esta con respecto a su media. En un valor promedio se puede apreciar una similitud con la distribución Normal o Gaussiana \cite{kurt}.

\pagebreak

\begin{figure*}
    \includegraphics[width=\textwidth]{filters}
    \caption{\it Código utilizado en la formulación de los filtros simbólicos basados en conocimiento.}
\end{figure*}

\begin{figure*}
    \includegraphics[width=\textwidth]{filter_bank}
    \caption{\it Código utilizado para crear la matriz de Filtros Espejados para el proceso de downsampling.}
\end{figure*}

\begin{figure*}
    \includegraphics[width=\textwidth]{transform}
    \caption{\it  Código utilizado para realizar la transformación de nivel medio a la señal de entrada, utilizando el algoritmo planteado en el {\bf Subproblema 2}}
\end{figure*}

\clearpage

\section*{Bibliografía}

\bibliographystyle{plain}
\bibliography{refs}

\end{document}